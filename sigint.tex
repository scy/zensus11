\documentclass[ignorenonframetext,ucs]{beamer}

\usepackage[ngerman]{babel}
\usepackage{ucs}
\usepackage[utf8x]{inputenc}
\usepackage[T1]{fontenc}
\usepackage{relsize}
\usepackage{lmodern}

\PreloadUnicodePage{0}

%\usepackage[colorlinks]{hyperref}
\newcommand\refurl[2]{#2\footnote{\url{#1}}}

\usepackage{graphicx}
\definecolor{darkgreen}{rgb}{0,0.5,0}
\usetheme[compress]{Berlin}

\title{Unter~dem~Radar~— Das~Zensusgesetz~2011}
\subtitle{All~Your~Database Are~Belong To~Us}
\author{Oliver~„Unicorn“~Knapp \and Tim~„Scytale“~Weber}
\institute{
	Chaos~Computer~Club~e.V. \and
	oqlt~e.V. \and
	RaumZeitLabor~Mannheim
}
\date{22.~Mai~2010\\SIGINT~2010, Köln}

\begin{document}

\frame{\titlepage}

\section{Einführung}

\subsection{Die Referenten}

\subsection{Der Zensus, das unbekannte Wesen}

\begin{frame}{Was ist ein Zensus?}\begin{itemize}
\item Volkszählung, Zensus, Census
\item mehr als nur Zählen
\item Makrozensus: alle werden gezählt
\item Mikrozensus: repräsentative Stichproben
\item Durchführungsarten:\begin{itemize}
	\item „klassischer“ Zensus: Umfragebögen
	\item Registerzensus: keine Umfragen, sondern Melderegister
	\item registergestützter Zensus: Mischform;\\Melderegister ergänzt durch Umfragen
\end{itemize}
\end{itemize}\end{frame}

\begin{frame}{Warum das Ganze?}\begin{itemize}
\item Bevölkerungszusammensetzung und -verteilung
\item Planung (Wohnungsbau, Infrastruktur, …)
\item Finanzierung (Steuerschätzung, Länderfinanzausgleich,\\kommunale Haushalte, …)
\item Bildungsniveau, Ausländeranteil, Religionszugehörigkeit, Familienstruktur, …)
\end{itemize}\end{frame}

\subsection{Geschichte der Volkszählungen}

\begin{frame}{Im Nationalsozialismus}\begin{itemize}
\item 1933, 
\end{itemize}\end{frame}

\subsection{Volkszählungen im Nationalsozialismus}

\subsection{Volkszählungen in der DDR}

\subsection{Die Volkszählung 1983/87}

\subsection{Volkszählungen nach der Wiedervereinigung}

\begin{frame}{Deutsche Volkszählung 1991}\begin{itemize}
\item
\end{itemize}\end{frame}

\begin{frame}{EU-Zensusrunde 2000/2001}\begin{itemize}
\item
\end{itemize}\end{frame}

\begin{frame}{Steuernummer 2007}\begin{itemize}
\item
\end{itemize}\end{frame}

\section{Volkszählung 2011}

\subsection{EU-Zensusrunde}

\subsection{Beschluss auf Bundesebene}

\subsection{Umsetzung in den Ländern}

\section{Nächste Schritte}

\subsection{Verfassungsklage}

\subsection{Bevölkerung informieren}

\end{document}
